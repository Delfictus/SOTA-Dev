\documentclass{article}
\usepackage{graphicx}
\usepackage{hyperref}

\title{NiV-Bench: A Neuromorphic Benchmark for Cryptic Epitope Prediction on Nipah Virus Proteins}

\author{[Authors]}

\begin{document}

\maketitle

\begin{abstract}
Cryptic epitopes—antibody binding sites hidden in the apo conformation—
represent high-value therapeutic targets that evade current prediction methods.
We introduce NiV-Bench, a benchmark for evaluating cryptic site prediction
on Nipah virus proteins, a WHO priority pathogen with no approved treatments.
Using PRISM>4D, a GPU-accelerated pipeline integrating neuromorphic 
computing with reinforcement learning, we achieve [X]\% improvement in 
cryptic site detection over EVEscape while processing structures 
19,400\times faster. Our FluxNet RL agent learns optimal feature weighting
from a 6D state space derived from topological, physics-based, and
spiking neural network features.
\end{abstract}

\section{Introduction}
% 1. Nipah virus threat (>70% fatality, no drugs)
% 2. Cryptic epitopes as therapeutic targets
% 3. EVEscape's limitations (sparse class 4 coverage)
% 4. Our contribution: neuromorphic + RL for cryptic sites

\section{Methods}
% 2.1 NiV-Bench Dataset
% 2.2 PRISM>4D Architecture (47 CUDA kernels, 140-dim features)
% 2.3 FluxNet RL (6D state space, Q-learning)
% 2.4 Evaluation Metrics

\section{Results}
% 3.1 Cryptic Site Detection (Task 1)
% 3.2 Epitope Prediction (Task 2)
% 3.3 ΔΔG Prediction (Task 3)
% 3.4 Speed Benchmark (Task 4)
% 3.5 Ablation Studies

\section{Discussion}
% 4.1 Why neuromorphic features matter
% 4.2 Clinical implications for Nipah therapeutics
% 4.3 Limitations and future work

\section{Conclusion}

\end{document}
